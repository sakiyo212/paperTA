% Mengubah keterangan `Abstract` ke bahasa indonesia.
% Hapus bagian ini untuk mengembalikan ke format awal.
\renewcommand\abstractname{Abstrak}

\begin{abstract}

  Diabetic retinopathy (DR) is a microvascular complication of diabetes and is the leading cause of blindness among working-age adults worldwide. Early detection and intervention are crucial to prevent vision loss and improve patient outcomes. However, traditional screening methods often face limitations in accuracy and accessibility. This study proposes the implemen-tation of a Residual Neural Network (ResNet) for automated DR detection and classification from fundus images. By achieving these objectives, this study aims to contribute to the advance-ment of automated DR diagnosis and ultimately improve patient care through early intervention and personalized treatment strategies.The most accurate model, ResNet-18, achieved the best validation accuracy without any adjustment to the class weight, with a value of 0.8211. Addi-tionally, the model with the highest Kappa score was ResNet-18, which had the best training accuracy without any modification to the class weight, resulting in a Kappa score of 0.7584.

\end{abstract}

% Mengubah keterangan `Index terms` ke bahasa indonesia.
% Hapus bagian ini untuk mengembalikan ke format awal.
\renewcommand\IEEEkeywordsname{Keywords}

\begin{IEEEkeywords}

  % Ubah kata-kata berikut sesuai dengan kata kunci dari penelitian.
  Diabetic retinopathy, ResNet, Deep Learning, Optical Coherence Tomography Angiography Analysis.

\end{IEEEkeywords}
