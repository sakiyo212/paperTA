% Ubah judul dan label berikut sesuai dengan yang diinginkan.
\section{Introduction}
\label{sec:pendahuluan}

% Ubah paragraf-paragraf pada bagian ini sesuai dengan yang diinginkan.

In the present age of digital technology, artificial intelligence has become intimately intertwined with several facets of human life. These advancements in technology have had a positive impact on various aspects of our lives. They have enhanced productivity by providing content recommendations on social media, virtual assistants, and spam filters. Additionally, they have improved efficiency through the implementation of intelligent transportation systems and automatic scheduling. Furthermore, they have contributed to entertainment and have played a significant role in the research and development sector of science and technology.
 
Diabetic retinopathy is a consequence of diabetes mellitus (DM) that occurs when blood vessels in the retina are damaged. The condition can result in diminished eyesight and potentially complete loss of vision (Yusran, 2022).

Approximately 9.3 million individuals worldwide are afflicted with blindness caused by diabetic retinopathy, as reported by the World Health Organization (WHO). The projected figure is anticipated to rise to 12.6 million by the year 2040.

Early detection of diabetic retinopathy is essential in order to avoid the advancement of the illness and minimize the likelihood of severe consequences. Technology, particularly in the field of medical image processing, is now essential in the medical industry to enhance the early diagnosis of diabetic retinopathy. The Residual Neural Network (ResNet) is a well-established approach with numerous tools available for in-depth study.

ResNet is a neural network specifically created to tackle the issue of declining performance in neural networks with more depth. The residual technique enables ResNet to effectively optimize network learning on intricate data, such as medical images. The objective of this study is to examine diabetic retinopathy using a Convolutional Artificial Neural Network. 


Pembahasan pada paper ini dimulai dengan presentasi mengenai penelitian lain (Bagian \ref{sec:penelitianterkait}).
Kemudian dilanjutkan dengan penjelasan mengenai arsitektur dari sistem yang dibuat (Bagian \ref{sec:arsitektur}).
Berdasarkan hal tersebut, kami menunjukkan lorem ipsum (Bagian \ref{sec:loremipsum}).
Terakhir, didapatkan kesimpulan dari penelitian yang telah dilakukan (Bagian \ref{sec:kesimpulan}).
