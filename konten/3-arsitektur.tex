% Ubah judul dan label berikut sesuai dengan yang diinginkan.
\section{Arsitektur}
\label{sec:architecture}

% Ubah paragraf-paragraf pada bagian ini sesuai dengan yang diinginkan.

\subsection{Dataset Penelitian}
\label{subsec:dataset}

The data utilized in this study were obtained from the Diabetic Retinopathy Analysis Grand Challenge, in the form of OCT-Angiography images.
A table, referenced as Table \ref{table:Datasettraining} in the text, provides a detailed overview of the data distribution utilized in this study.
\begin{figure}[hbtp]
	\centering
	\subfloat[\centering Non-DR]{{\includegraphics[width=2.5cm]{gambar/non-DR.png} }}%
	\subfloat[\centering NPDR]{{\includegraphics[width=2.5cm]{gambar/NPDR.png} }}%
	\subfloat[\centering PDR]{{\includegraphics[width=2.5cm]{gambar/PDR.png} }}%
	\caption{Contoh Data Citra Retina}
	\label{fig:sampleDataset}
\end{figure}

The image provided is a retinal fundus photograph, which is utilized for the evaluation of diabetic retinopathy. The image depicts the intricate details of the retinal vascular network, the optic disk, and the macular area. The following section provides a comprehensive account of the image in question.

Dataset Information from the DRAC Challenge
The dataset utilized for training and testing comprises the following:

	\begin{itemize}
		\item Training Set: 611 gambar
		\item Testing Set: 386 gambar
	\end{itemize}
Subsequently, the images in the training set are divided into two subsets, namely the training set and the validation set, with the number of images in each subset determined in accordance with the specifications outlined in Table \ref{table:Datasettraining}.
\begin{table}[hbtp]
	\begin{center}
	\caption{Table of distribution Set for Training and Validation}
	\label{table:Datasettraining}
	\begin{tabular}{|l|l|l|l|}
		\hline
		\rowcolor[HTML]{C0C0C0} 
		Label                                                & Classification & Amount & Total                                         \\ \hline
		\rowcolor[HTML]{FFFFFF} 
		\cellcolor[HTML]{FFFFFF}                             & non-DR      & 263    & \cellcolor[HTML]{FFFFFF}                      \\ \cline{2-3}
		\rowcolor[HTML]{FFFFFF} 
		\cellcolor[HTML]{FFFFFF}                             & NPDR        & 169    & \cellcolor[HTML]{FFFFFF}                      \\ \cline{2-3}
		\rowcolor[HTML]{FFFFFF} 
		\multirow{-3}{*}{\cellcolor[HTML]{FFFFFF}Training}   & PDR         & 56     & \multirow{-3}{*}{\cellcolor[HTML]{FFFFFF}488} \\ \hline
		\rowcolor[HTML]{FFFFFF} 
		\cellcolor[HTML]{FFFFFF}                             & non-DR      & 66     & \cellcolor[HTML]{FFFFFF}                      \\ \cline{2-3}
		\rowcolor[HTML]{FFFFFF} 
		\cellcolor[HTML]{FFFFFF}                             & NPDR        & 43     & \cellcolor[HTML]{FFFFFF}                      \\ \cline{2-3}
		\rowcolor[HTML]{FFFFFF} 
		\multirow{-3}{*}{\cellcolor[HTML]{FFFFFF}Validation} & PDR         & 14     & \multirow{-3}{*}{\cellcolor[HTML]{FFFFFF}123} \\ \hline
		\end{tabular}
	\end{center}
\end{table}

The images in the test set are employed to obtain an online assessment using the Quadratic Weighted Kappa metric, with the model from DRAC serving as a point of comparison. However, due to the absence of a label in the given test set, it is not feasible to utilize this set for the calculation of other metrics, such as precision, recall, and F1-score.

\subsection{Methodology}
\label{subsec:loremipsum}

Perbandingan metode skenario yang digunakan dalam pelatihan model ResNet:
Dalam penelitian ini, dikarenakan oleh dataset yang sedikit dan ada class yang kurang representatif, dilakukan beberapa metode untuk penyeimbangan dataset.
\begin{itemize}
	\item Default
	
	Tidak ada Tindakan yang dilakukan untuk menyeimbangkan dataset. Metode ini dilakukan untuk variabel kontrol
	\item Penyesuaian \emph{Class-weight}
	
	Pada metode ini, dilakukan penambahan weight agar class yang underrepresented memiliki beban lebih tinggi
\end{itemize}

\begin{figure}[hbtp] \centering
	% Nama dari file gambar yang diinputkan
	\includegraphics[scale=0.5]{gambar/diagramMethod.png}
	% Keterangan gambar yang diinputkan
	\caption{Diagram blok metodologi}
	% Label referensi dari gambar yang diinputkan
	\label{fig:diagramMethod}
\end{figure}

\subsection{Pelatihan Model}
\label{sec:325}
The training of the model is conducted using the transfer learning method. The transfer learning method is performed using the ResNet model, which has been selected and trained with the ImageNet dataset.
The model architecture utilized is as described in \ref{subsec:loremipsum}, with the hyperparameters specified in Table \ref{tb:hyperParameterTraining}.
\begin{table}[hbtp]
	\begin{center}
		\caption{Hyperparameter}
		\label{tb:hyperParameterTraining}
		\begin{tabular}{|
		>{\columncolor[HTML]{C0C0C0}}l |l|lll}
		\cline{1-2}
		Input shape                                         & 224,224,3      &  &  &  \\ \cline{1-2}
		Opimizer                                            & Adam           &  &  &  \\ \cline{1-2}
		Loss Function                                       & Cross Entropy  &  &  &  \\ \cline{1-2}
		Learning Rate                                       & 0.1            &  &  &  \\ \cline{1-2}
		Momentum                                            & 0.9            &  &  &  \\ \cline{1-2}
		\cellcolor[HTML]{C0C0C0}                            & Step size = 10 &  &  &  \\ \cline{2-2}
		\multirow{-2}{*}{\cellcolor[HTML]{C0C0C0}Scheduler} & Gamma = 0.1    &  &  &  \\ \cline{1-2}
		Epoch                                               & 100            &  &  &  \\ \cline{1-2}
		Batch size                                          & 32             &  &  &  \\ \cline{1-2}
		\end{tabular}
	\end{center}
\end{table}

