% Ubah judul dan label berikut sesuai dengan yang diinginkan.
\section{Penelitian Terkait}
\label{sec:penelitianterkait}

\subsection{\emph{Classification of Diabetic Retinopathy Based on B-ResNet}}
\label{subsec:penelitianTerdahulu1}
Zhang dan rekan \citet{zhang2022residual} pada penelitiannya menggunakan data set Eye-PACS, MESSIDOR-2, dan IDRiD untuk membangun data set DR dengan pembersihan, penguatan, dan normalisasi gambar. Selain itu, digunakan metode prapemrosesan gambar yang ditingkatkan untuk meningkatkan fitur gambar fundus. Model B-ResNet dibangun dengan menggabungkan keunggulan ekstraksi fitur ResNet50 dan fusi fitur BCNN. Selain itu, sebelum fusi fitur, gambar fitur yang diekstraksi oleh ResNet50 diproses oleh modul perhatian saluran. ResNet50 dipralatih pada data set ImageNet dan parameternya di-fine-tune melalui transfer learning.

Hasil penelitian menunjukkan bahwa model B-ResNet mencapai akurasi 71,11\% , ACA 0,714, Kappa 0,634, dan macro-F1 0,711. Hasil ini lebih tinggi daripada penelitian sebelumnya. Percobaan perbandingan membuktikan bahwa metode prapemrosesan gambar yang ditingkatkan meningkatkan akurasi, ACA, Kappa, dan nilai macro-F1 model.

\subsection{\emph{A Deep Learning Framework for Detection and Classification of Diabetic Retinopathy in FundusImages Using Residual Neural Networks}}
\label{subsec:penelitianTerdahulu2}
Abini dan rekan \citet{10335079} melakukan studi menggunakan model ResNet, yang dilatih dengan dataset APTOS, untuk melakukan klasifikasi biner dan multikelas menggunakan jaringan saraf konvolusional dalam (deep convolutional neural network). Hasil eksperimen menunjukkan bahwa model dengan lapisan dalam seperti ResNet-50 dapat meningkatkan kinerja keseluruhan dataset. Ini mengindikasikan bahwa penggunaan model ResNet-50 dalam klasifikasi DR dapat menjadi lebih efisien dalam hal waktu, tenaga kerja, dan biaya dibandingkan dengan metode diagnostik manual.
