% Ubah judul dan label berikut sesuai dengan yang diinginkan.
\section{Related Research}
\label{sec:relatedresearch}

\subsection{\emph{Classification of Diabetic Retinopathy Based on B-ResNet}}
\label{subsec:penelitianTerdahulu1}
 
Zhang and colleagues \citet{zhang2022residual} utilized the Eye-PACS, MESSIDOR-2, and IDRiD datasets to construct the DR dataset through image cleaning, enhancement, and normalization. Additionally, an enhanced image preprocessing method was employed to enhance the fundus image features. The B-ResNet model was developed by integrating the strengths of ResNet50 feature extraction and BCNN feature fusion. Furthermore, prior to feature fusion, the extracted feature images from ResNet50 undergo processing by the channel attention module. ResNet50 is trained on the ImageNet dataset and its parameters are fine-tuned through transfer learning.

The results demonstrated that the B-ResNet model achieved an accuracy of 71.11\%, ACA of 0.714, Kappa of 0.634, and macro-F1 of 0.711. These results exceed those of previous studies. The comparison experiment demonstrated that the enhanced image preprocessing method enhanced the accuracy, ACA, Kappa, and macro-F1 values of the model.

\subsection{\emph{A Deep Learning Framework for Detection and Classification of Diabetic Retinopathy in FundusImages Using Residual Neural Networks}}
\label{subsec:penelitianTerdahulu2} 
Abini and colleagues \citet{10335079} conducted a study using the ResNet model, trained on the APTOS dataset, to perform binary and multiclass classification using deep convolutional neural networks. The experimental results demonstrate that models with deep layers, such as ResNet-50, can enhance the overall performance of the dataset. This suggests that the utilization of ResNet-50 models in DR classification can be more expedient, cost-effective, and labor-saving compared to manual diagnostic methods.
